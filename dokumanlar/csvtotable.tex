\documentclass{article}

\usepackage{lipsum}
\usepackage{filecontents}
\usepackage{booktabs}
\usepackage{array}
 \usepackage{pgfplots}
 \pgfplotsset{compat=1.7}
\usepackage{pgfplotstable}
\pgfplotstableset{
  col sep=comma,
  every head row/.style={%
    before row=\toprule,%
    after row=\midrule},%
  every last row/.style={%
    after row=\bottomrule}%
}

\begin{filecontents}{firstTable.csv}
A,B,C,D,E
1,2,3,4,5
{$\Sigma$i},ii,iii,iv,v
\end{filecontents}

\begin{filecontents}{secondTable.csv}
a,b,c,d,e
1,2,3,4,5
I,II,III,IV,V
\end{filecontents}

\begin{filecontents}{section1.tex}
  Here's a nice table:

  \pgfplotstableread{firstTable.csv}{\loadeddataone}

  \begin{table}
    \centering
    \pgfplotstabletypeset[%
       columns={A,B,C,D,E},
       string type
       ]{\loadeddataone}
    \caption{Simple caption for a simpler table.}
    \label{tab:table1}
  \end{table}
\end{filecontents}


\begin{document}
\section{Here goes section 1}
\lipsum[1]
\include{section1}


\section{Section 2}
And here we put another table.
 \pgfplotstableread{secondTable.csv}{\loadeddatatwo}

 \begin{table}
   \centering
   \pgfplotstabletypeset[%
     columns={a,b,c,d,e},
     string type
     ]{\loadeddatatwo}
     \caption{Fancy caption for a simple table.}
     \label{tab:table2}
 \end{table}



\end{document}