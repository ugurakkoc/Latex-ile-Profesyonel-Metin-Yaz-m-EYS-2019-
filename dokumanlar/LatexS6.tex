\documentclass[pdf]{beamer}     
\usepackage[turkish]{babel}
\usepackage[utf8]{inputenc}
\usepackage{subcaption}
\usepackage{tikz}
\usepackage{amsmath}
\usetheme{default} 
\usepackage{graphicx}                            
\useinnertheme{default}
\usefonttheme{professionalfonts}
\usecolortheme{default}
\title{Latex ile Profesyonel Metin Yazımı-6}  
\author{Uğur AKKOÇ}      
\begin{document}  
	\shorthandoff{=}
	\maketitle
	
	
\begin{frame}{Referans Yönetimi}
\begin{itemize}
	\item Latex referans yönetimi için kullanışlı bir sistem sunuyor.
	\item Latex ile hazırlanan dosyalarda atıf girmek ve rferans listesi dökmek için .bib uzantılı bir dosya kullanılır.
	\item Referans yönetimi için, tıpkı kaynak kod dosyası gibi referans listesinde yer alan kaynakların bilgilerinin girildiği bir bib dosyası hazırlanır.
\end{itemize}
\end{frame}

\begin{frame}{Referans Yönetimi}
\begin{itemize}
	\item bib dosyasını TeXstudio (ya da diğer metin edtörlerinde) hazırlayıp, .bib uzantısı ile kaydedebilirsiniz.
	\item Kaynak dosyasında kaydettiğiniz isim ile çağırdığınızda referans dosyanız okunacaktır.
	\item Referans dosyanız olan bib dosyası da mutlaka diğer dosyalarınız ile aynı klasörde olmalıdır.
\end{itemize}
\end{frame}

\begin{frame}{Referans Yönetimi}
\begin{itemize}
	\item bib dosyası üretmek için bir kaç yöntem kullanılabilir:
	\begin{enumerate}
		\item Metin editöründe oluşturulan bib dosyasındaki bilgileri elle girmek
		\item Metin editöründe oluşturulan bib dosyasındaki bligileri Google Scholar' dan almak
		\item Herhangi bir makalenin dergi sayfasından bib uzantısını almak
		\item Bir kaynaktan bib dosyası üreten generator kullanmak
		\item Mendeley gibi bir referans yönetimi programı ile grupladığınız makalelerin bib dosyasını toplu olarak almak
	\end{enumerate}
\end{itemize}
\end{frame}

\begin{frame}{Referans Yönetimi}
\begin{itemize}
	\item bib dosyası oluşturulduktan sonra, kaynak dosyanızda:
	\begin{itemize}
		\item Metin içinde atıf vermek için cite komutu kullanılır.
		\item Referans listesi ise bibliography komutu ile hazırlanır.
		\item bibliographystyle komutu ise kaynak gösterme biçimini belirler.
	\end{itemize}
\end{itemize}
\end{frame}

\begin{frame}
apacite paketi kullanılır. \\
\begin{figure}
	\includegraphics{refe1.jpg}
\end{figure}
\end{frame}

\begin{frame}
ya da natbibapa sçeeneği ile apacite paketi kullanılır. \\
\begin{figure}
	\includegraphics{refe2.jpg}
\end{figure}
\end{frame}

\end{document}
	