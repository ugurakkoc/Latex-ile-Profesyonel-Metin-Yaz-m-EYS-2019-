\documentclass[pdf]{beamer}     
\usepackage[turkish]{babel}
\usepackage[utf8]{inputenc}
\usepackage{subcaption}
\usepackage{tikz}
\usepackage{amsmath}
\usetheme{default}                             
\useinnertheme{default}
\usefonttheme{professionalfonts}
\usecolortheme{default}
\title{Latex ile Profesyonel Metin Yazımı-5}  
\author{Uğur AKKOÇ}      
\begin{document}  
	\shorthandoff{=}
	\maketitle
	
	
\begin{frame}{Özet Ortamı}
	begin{abstract} ve end{abstract} komutları arasında girilen metin Latex tarafından özet olarak algılanır. Bu bölüm içindekiler kısmında özet bölümü olarak isimlendirilir.
\end{frame}

\begin{frame}{Grafik Ortamı}
\begin{itemize}
	\item Grafik eklemek için kullanılır.
	\item Öncelikle, metninize halihazırda kayıtlı bir resim ya da grafik eklemek için:
	\begin{enumerate}
		\item graphicx paketini
		\item figure ortamını
		\item includegraphics komutunu
	\end{enumerate}
kullanmalısınız.
\end{itemize}
\end{frame}

\begin{frame}{Grafik Ortamı - Dikkat Edilmesi Gerekenler}
\begin{itemize}
	\item Metne ekleyeceğiniz grafik, kaynak kod dosyanızın olduğu ve çıktı dosyanızın derlendiği klasörün içinde olmalıdır.
	\item jpg, png vb. dosya türleri eklenebilir. Ancak çözünürlük yeterli olduğu sürece en düşük dosya boyutu olan dosya türünü seçmek derleme işleminin kısa sürmesi açısından avantajlıdır.
	\item Tıpkı tablolarda olduğu gibi, caption, label ve ref komutları ek olarak çalıştırılabilir.
\end{itemize}
\end{frame}


\begin{frame}{Grafik Ortamı - Grafik Pozisyonu Ayarlama}
\begin{itemize}
	\item begin{figure} komutundan sonra köşeli parantez içerisinde aşağıdaki seçenekler ile pozisyonlama yapılabilir:
		\begin{enumerate}
			\item h: here
			\item t: top of page
			\item b: bottom
			\item p: on new page
			\item !
		\end{enumerate}
\end{itemize}
\end{frame}

\begin{frame}{Grafik Ortamı - Birden Fazla Grafik}
\begin{itemize}
	\item Bazı durumlarda birden fazla grafik eklemek gerekebilir. Bunun için subcaption paketi ve subfigure ortamı kullanılabilir.
	\item Grafik girdisi yine includegraphics komutu kullanılabilir.
	\item Birden çok grafik eklenmek istendiğinde, grafiklerin boyutunu ayarlamak daha önemli hale geliyor. includegraphics komutunun sonuna küme parantezi ile ayar girilebilir. Örneğin, 0.4linewidth gibi.
\end{itemize}
\end{frame}

\begin{frame}{Grafik Ortamı - Bir Örnek}
\begin{figure}
	\includegraphics[width=\linewidth]{boat.png}
	\caption{A boat.}
	\label{fig:boat1}
\end{figure}

Figure \ref{fig:boat1} shows a boat.
\end{frame}

\begin{frame}{Grafik Ortamı - Bir Örnek}
\begin{figure}[h!]
	\centering
	\begin{subfigure}[b]{0.4\linewidth}
		\includegraphics[width=\linewidth]{boat.png}
		\caption{Coffee.}
	\end{subfigure}
	\begin{subfigure}[b]{0.4\linewidth}
		\includegraphics[width=\linewidth]{boat.png}
		\caption{More coffee.}
	\end{subfigure}
	\caption{The same cup of boat. Two times.}
	\label{fig:boat2}
\end{figure}
\end{frame}

\begin{frame}{Grafik Ortamı - tikz}
\begin{itemize}
	\item Daha karmaşık şekiller çizebilmek için tikz paketi kullanılabilir. Komut olarak draw[tikz] kullanılıyor.
\end{itemize}
\end{frame}

\begin{frame}{Grafik Ortamı - tikz}
\begin{figure}[h!]
	\begin{center}
		\begin{tikzpicture}
		\draw [red,dashed] (-2.5,2.5) rectangle (-1.5,1.5) node [black,below] {Start}; % Draws a rectangle
		\draw [thick] (-2,2) % Draws a line
		to [out=10,in=190] (2,2)
		to [out=10,in=90] (6,0) 
		to [out=-90,in=30] (-2,-2);    
		\draw [fill] (5,0.1) rectangle (7,-0.1) node [black,right] {Obstacle}; % Draws another rectangle
		\draw [red,fill] (-2,-2) circle [radius=0.2] node [black,below=4] {Point of interest}; % Draws a circle
		\end{tikzpicture}
		\caption{tikz paketi ile örnek şekil}
	\end{center}
\end{figure}
\end{frame}

\begin{frame}{Matematik Ortamı}
\begin{itemize}
	\item Latex programında matematiksel ifadeleri girmek için iki yöntem vardır:
	\begin{enumerate}
		\item Satır içinde matematiksel ifadeleri girmek
		\item Matematik ortamı açarak matematiksel eşitlikleri girmek
	\end{enumerate}
\end{itemize}
\end{frame}

\begin{frame}{Matematik Ortamı - Satır İçi}
\begin{itemize}
	\item Satır içinde matematiksel ifadeleri kullanmak için basit komutlar yeterlidir. Metnin içindeki herhangi bir matematiksel ifadeyi iki adet dolar işareti arasına alırsak, Latex bu ifadeyi matematiksel ifade olarak doğru biçimde dizecektir.
	\item Satır içi matematiksel ifadelere bir örnek olarak, $\alpha + \beta = 1$ ifadesi gösterilebilir.
\end{itemize}
\end{frame}

\begin{frame}{Matematik Ortamı - Matematik Ortamı}
\begin{itemize}
	\item amsmath paketi kullanılır. Bu paket sahanlıkta kullanıldıktan sonra, begin{equation} komutu ile denklem girilir.
	\item Bu yapıda, tablo ve şekillere benzer şekilde label ve ref komutları kullanılabilir.
\end{itemize}
\end{frame}

\begin{frame}{Matematik Ortamı -Örnek}
Ölçeğe göre artan getirili bir üretim fonksiyonunun koşulu: $\alpha + \beta > 1$ \newline
\begin{equation} \label{denklem1}
F(\sin x, \cos x)=F_1(\cos x) + \sin x F_2(\cos x)
\end{equation} \newline

\ref{denklem1} nolu denklem trigonometrik bir fonksiyondur.
\end{frame}

\end{document}
	