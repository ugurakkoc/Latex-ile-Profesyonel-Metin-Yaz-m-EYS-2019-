\documentclass[a4paper, 12pt]{article}
\usepackage[turkish]{babel}
\usepackage[utf8]{inputenc}
\usepackage{setspace}
\usepackage{hyperref}

\begin{document}
	\section{Temel Komutlar için Örnek İfadeler}
            \textbf{Satır} başında boşluk eklerseniz, Latex bu boşlukları dizmez.
           
Benzer şekilde,            birden fazla boşluk eklemek sonucu değiştirmiyor.

\nonfrenchspacing Latex metnin içinde noktadan sonra otomatik olarak boşluk bırakmaktadır. İlgili komut ile bu ayar değiştirilebilir.

\frenchspacing Yazar dil paketinde belirtilenden daha farklı bir heceleme önerisinde bul\-unm\-ak isterse, ilgili kelimenin istediği yerlerine tire koyarak bunu gerçekleştirebilir.

\doublespacing Benzer şekilde bir satırın \newline belirli bir sözcükten sonra kesilmesi mümkündür.

\subsection{Metin Vurguları}
\textbf{Kalın} \newline
\textit{İtalik} \newline
\underline{Altı Çizili} \newline
\textsl{Eğik} \newline

\subsection{Metin Boyutu Değiştirme}
\tiny tiny \small small \normalsize normalsize \large large \Large Large \LARGE LARGE \footnotesize footnotesize \huge huge \Huge Huge \normalsize

\end{document}

